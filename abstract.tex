% abstract

\documentclass[main.tex]{subfiles}
\begin{document}
\setcounter{page}{1}
\chapter*{Abstract}
Fused Filament Fabrication (FFF) is arguably the most widely available Additive Manufacturing technology at the moment. Offering the possibility of producing complex geometries in a compressed product development cycle and in a plethora of materials, it comes as no surprise that FFF is attractive to multiple industries, including the automotive and aerospace segments. However, the high anisotropy of parts developed through this technique implies that part failure prediction is extremely difficult \textemdash a requirement that must be satisfied to guarantee the safety of the final user. Application of a Failure Criterion to predict part failure can solve this issue. However, a large number of mechanical tests performed under a variety of loading conditions are required to populate the parameters of the function that describes the failure envelope - a process that is extremely time consuming. This research proposal describes a method by which the development of the failure surface can be streamlined, and the number of mechanical tests can be significantly reduced. 
 
\vspace{10mm} %10mm vertical space
\textbf{Keywords:} FFF, FDM, Failure Criteria, Mechanical Testing, Machine Learning.

\vfill %Send copyright notice to bottom of the page
\begin{center}
Copyright~\textcopyright: Gerardo A. Mazzei Capote (2020)

\emph{All rights reserved}	
\end{center}
\end{document}