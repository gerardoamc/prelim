% 02_Inputs.tex
% the packages are put in a seperate file to help keep the main.tex file clean

% To use the fontec package properly make sure the cm-super
% package is installed for MikTex.
% https://docs.miktex.org/2.9/manual/pkgmgt.html
\usepackage[T1]{fontenc}
\usepackage{fancyhdr}

\usepackage[textwidth=65pt]{todonotes}
\usepackage{siunitx} % provides formating for numbers with SI units

\usepackage{subfiles} % For being able to compile the subfiles on their own

%For math related stuff
\usepackage{amsmath}
\usepackage{systeme}

\usepackage{lipsum} % Used for making dummy text

\usepackage{booktabs} % Makes tables better
\usepackage{tablefootnote} % for making footnotes in tables

% used for loading graphics
\usepackage{graphicx}
\graphicspath{{Figures/}}
% used for making sub-figures
\usepackage{subfig}
%used for placing text over images
\usepackage[percent]{overpic}

% Helps with making arrow with overpic
\usepackage{pict2e}

% use the \layout command and compile to show margins
\usepackage{layout}

% For figures with text wrapped around them
\usepackage{wrapfig}
% For tables with longer text, where manipulating space may be a pain
\usepackage{tabu}
% For special formating of lists
\usepackage{enumitem}

\usepackage{nomencl}
\makenomenclature
\renewcommand{\nomname}{Symbols and Acronyms}

% From https://www.sharelatex.com/learn/Nomenclatures
\usepackage{etoolbox}
\renewcommand\nomgroup[1]{%
  \item[\bfseries
  \ifstrequal{#1}{S}{Symbols}{%
  \ifstrequal{#1}{A}{Acronyms}}%
]}
 
% This will add the units to nomenclature
%----------------------------------------------
\newcommand{\nomunit}[1]{%
\renewcommand{\nomentryend}{\hspace*{\fill}#1}}
%----------------------------------------------

\usepackage{titlesec}

\titleformat{\chapter}
  {\huge\bfseries} % format
  {\thechapter \enspace}   % label
  {0pt}             % sep
  {\huge}           % before-code

% Change the plain style used by chapters as shown on page 7 of
% the fancyhdr manual: http://texdoc.net/texmf-dist/doc/latex/fancyhdr/fancyhdr.pdf
\fancypagestyle{plain}{%
\fancyhf{} % clear all header and footer fields
\fancyhead[LE, RO]{\thepage} %RO=right odd, RE=right even
\renewcommand{\headrulewidth}{0pt}
\renewcommand{\footrulewidth}{0pt}}

\setlength{\headheight}{14.5pt} % to prevent the \headheight warning
\fancyhf{}

%%% Use these four to make it look like a nice book
\fancyhead[LE, RO]{\thepage} % Page number
\fancyhead[LO, RE]{\slshape \leftmark} % Chapter number and name
\renewcommand{\chaptermark}[1]{\markboth{\thechapter.\ #1}{}}
\renewcommand{\headrulewidth}{1pt}

%%% Use these two make it fit the University Guidelines
%\fancyhead[RE, RO]{\thepage}
%\renewcommand{\headrulewidth}{0pt}


% set the margins to the UW-Madison's standard
\usepackage[left=1.3in,
                        top=1.3in,
                        right=1.1in,
                        bottom=1.1in,
                        marginparwidth=65pt]
                        {geometry}

\usepackage[backend=bibtex, sorting=none,maxbibnames=99]{biblatex} %References are numbered per order of use in the text as opposed to alphabetically (default)
\addbibresource{BibTex/prelim.bib}

\usepackage[numbered]{matlab-prettifier} % Used to import MATLAB code 
\usepackage{epigraph} % Only used in the SciSlice chapter
%\usepackage{pdfpages} % Used to import pdfs onto the document

%------------------------------------------------------------------------------------		
%Hyperlinks and PDF Settings
\usepackage[
	bookmarksopen =false, 				% Display bookmarks when the document is opened
	pdftoolbar =true, 					% Display Acrobat reader toolbar
	bookmarksnumbered =true,			% Display section numbers
	pdfpagelabels = false, 				% Display original page numbers
	%plainpages = false, 				% Blank pages
	hyperfootnotes=true,
	pdfpagelayout = TwoPageRight, 		% Open PDF as 2 sided when opened	
]{hyperref}

%Hyperref format
\hypersetup{
	colorlinks=true,	% Hyperlinks are colored
	linkcolor=blue,		% Color of internal links (within document)
	citecolor=blue,	    % Color of internal links to the Reference page (within document)
	urlcolor=blue,		% Color of URLs (external) - default is magenta.
	pdftitle = {Predicting mechanical properties of FFF parts through Machine Learning},
	pdfsubject = {Prelim 2020},
	pdfauthor = {Gerardo A. Mazzei Capote},
	pdfkeywords = { },	
}
\usepackage{url}