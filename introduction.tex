% introduction.tex

\documentclass[main.tex]{subfiles}
\begin{document}
	%The next two lines ensure that the Introduction is a unnumbered chapter that shows up in the index
\chapter*{Introduction}\label{ch:intr}
\addcontentsline{toc}{chapter}{Introduction} \markboth{Introduction}{}
%Chapter body
\emph{Additive Manufacturing} (AM) is an umbrella term that encompasses all fabrication techniques where the final geometry of the part is obtained through superposition of material in a layer-by-layer basis \cite{Gibson2015}. Developed in the 1980s, this manufacturing technique permits immensely shorter part development cycles, since the transition from a 3D \emph{Computer Aided Design} (CAD) to part fabrication only requires one intermediate step: the use of a slicing engine that converts the geometry of the object into machine instructions \cite{Gibson2015}. For this reason, AM technologies were initially employed exclusively for prototype development and were referred to as \emph{Rapid Prototyping techniques} (RP). However, recent innovations in the field have caused AM to be considered as a legitimate manufacturing technology since it is also capable of reproducing complex geometries unattainable through traditional methods \cite{Gibson2015}.

While offering great advantages over traditional part fabrication methods, AM comes with its own set of limitations and disadvantages: First and foremost, the use of a stratified build approach tends to produce extremely anisotropic parts. Secondly, the geometric accuracy of the object produced is highly dependent of process parameters, particularly, the thickness of the layers. Finally, as of the time of this writing, AM lacks the standardization and scrutiny that are associated to most traditional manufacturing techniques \cite{Gibson2015}.  

\emph{Fused Filament Fabrication} (FFF), also known under the trademark \emph{Fused Deposition Modeling} (FDM\texttrademark), represents perhaps the most prevalent AM technique in the market due to the advent of low-cost, desktop 3D printers in the early 2010s \cite{Capote2017}. Due to the broad availability of machines and relatively low costs of material, there is a surging interest in optimizing FFF to produce small batches of end-user grade parts. Success stories are varied, but examples include vacuum form molds, fixtures, jigs, and tools used to aid assembly lines in the automotive industry \cite{Hartman2014, VanHulle2017,deVries2017}. However, this technology still faces the challenges and limitations that currently affect the field of AM as a whole. Namely, anisotropy introduced through the layer-by-layer build approach makes it difficult to assess the expected mechanical behavior of FFF parts when subjected to important mechanical stresses \cite{Capote2017}. For these reasons, multiple attempts have been made to characterize the anisotropy of FFF manufactured objects, such as the studies performed by Koch \emph{et al.} \cite{Koch2017} and Rankouhi \emph{et al.} \cite{Rankouhi2016}, which show that the ultimate tensile strength of FFF coupons is sensitive to process parameters such as the layer thickness and, in particular, the orientation in which the plastic strands are laid during the build process -henceforth referred to as the bead orientation. Literature related to preventing failure through predictive methods in the design stages is scarce. However, a handful of publications exist where this issue was solved through the application of a failure criterion. The reach of this methodology has been fairly limited, given the difficulty of using commercially available AM machines to produce test coupons with unconventional bead orientations necessary to populate the failure surface, as well as the limitations inherent to development of failure criteria. Examples include the developments of failure envelopes for \emph{Polyamide 12} (PA12) used in \emph{Selective Laser Sintering} (SLS) and Multi-Jet Fusion (MJF) \cite{Obst2018, Osswald2020}, and more importantly for this body of work, a failure surface for \emph{Acrylonitrile Butadiene Styrene} (ABS) used in FFF \cite{MazzeiCapote2019}. For the latter, certain test specimens in unconventional configurations had to be produced using a unique off-axis 3D printer developed in-house. In both cases, the researchers utilized a FC that incorporates stress interactions into the calculations of the failure surface, a feature that more recognized criteria, such as the Tsai-Wu model fail to take into account \cite{Osswald2017a}.

Additional predictive tools have been pushed to the forefront of engineering applications given recent developments in the fields of statistics, data science, artificial intelligence, worldwide connectivity, and computational hardware. These tools allow designing intelligent systems that can, among many things, detect and correct problems during a production run, identify trends, and more importantly for the objective of this work, predict outcomes or perform classification tasks. These tools have been grouped under the \emph{Machine Learning} (ML) moniker, and are currently being exploited by large companies to make sense of large clusters of data. Machine Learning tools thrive in cases where the inputs and outcomes of a particular phenomena or task are known, but connecting the two through an explicit set of rules or relationships can result extremely complex and time consuming \cite{Chollet2018} because, in simple terms, ML models are trained, as opposed to explicitly programmed. Their broad range of applications has caused its use to trickle into other segments of engineering, usually in the form of \emph{Neural Networks} or \emph{Support Vector Machines} performing a variety of regression analysis or classification tasks. The field of AM is no stranger to the ML topic. Interest in the subject has been remarked by several authors \cite{Qi2019, Razvi2019, Meng2020}, and it has even been successfully applied to predict certain properties of AM parts produced under various techniques \cite{Qi2019, Razvi2019, Meng2020, Sood2012}. 

The set of printing conditions that lead to an optimal part in terms of mechanical properties aren't still fully comprehended and result in extremely complex, multi-variable relations. However, an FFF machine with in-line sensors that allowed monitoring a variety of process-variables, as well as data generated from mechanical tests and ancillary experiments would constitute a perfect case for deployment of a Machine Learning system capable of predicting the mechanical properties of the finished part. In this manner, this work proposes to apply ML techniques to the FFF process in order to predict mechanical properties according to in-line measurements. Chapter \ref{ch:bg} will introduce basic concepts used throughout this work; Chapter \ref{ch:oocrit} will provide details pertaining to failure prediction of AM parts through failure criteria, focusing on previous work performed by the author on failure prediction for FFF parts; finally, Chapter \ref{ch:proposal} will supply information pertaining to why and how a ML solution is of interest, as well as displaying preliminary results available at the time of this writ.

%Nomenclature introduced in this chapter:
\nomenclature[A]{AM}{Additive Manufacturing}% 
\nomenclature[A]{RP}{Rapid Prototyping}%
\nomenclature[A]{CAD}{Computer Aided Design}%
\nomenclature[A]{FDM}{Fused Deposition Modeling\texttrademark}
\nomenclature[A]{FFF}{Fused Filament Fabrication}%
\nomenclature[A]{FC}{Failure Criterion}%
\nomenclature[A]{ABS}{Acrylonitrile Butadiene Styrene}%
\nomenclature[A]{PA12}{Polyamide 12}%
\nomenclature[A]{AI}{Artificial Intelligence}%

\end{document}