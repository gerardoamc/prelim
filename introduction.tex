% introduction.tex

\documentclass[main.tex]{subfiles}
\begin{document}
	%The next two lines ensure that the Introduction is a unnumbered chapter that shows up in the index
\chapter*{Introduction}
\addcontentsline{toc}{chapter}{Introduction} \markboth{Introduction}{}
%Chapter body
\emph{Additive Manufacturing} (AM) is an umbrella term that encompasses all fabrication techniques where the final geometry of the part is obtained through superposition of material in a layer-by-layer basis \cite{Gibson2015}. Developed in the 1980s, this manufacturing technique permits immensely shorter part development cycles, since the transition from a 3D \emph{Computer Aided Design} (CAD) to part fabrication only requires one intermediate step: the use of a slicing engine that converts the geometry of the object into machine instructions \cite{Gibson2015}. For this reason, AM technologies were initially employed exclusively for prototype development and were referred to as \emph{Rapid Prototyping techniques} (RP). However, recent innovations in the field have caused AM to be considered as a legitimate manufacturing technology since it is also capable of reproducing complex geometries unattainable through traditional methods \cite{Gibson2015}.

While offering great advantages over traditional part fabrication methods, AM comes with its own set of limitations and disadvantages: First and foremost, the use of a stratified build approach tends to produce extremely anisotropic parts. Secondly, the geometric accuracy of the object produced is highly dependent of process parameters, particularly, the thickness of the layers. Finally, as of the time of this writing, AM lacks the standardization and scrutiny that are associated to most traditional manufacturing techniques \cite{Gibson2015}.  

\emph{Fused Filament Fabrication} (FFF), also known under the trademark \emph{Fused Deposition Modeling} (FDM\texttrademark), represents perhaps the most prevalent AM technique in the market due to the advent of low-cost, desktop 3D printers in the early 2010s \cite{Capote2017}. Due to the broad availability of machines and relatively low costs of material, there is a surging interest in optimizing FFF to produce small batches of end-user grade parts. Success stories are varied, but examples include vacuum form molds, fixtures, jigs, and tools used to aid assembly lines in the automotive industry \cite{Hartman2014, VanHulle2017,deVries2017}. However, this technology still faces the challenges and limitations that currently affect the field of AM as a whole. Namely, anisotropy introduced through the layer-by-layer build approach makes it difficult to assess the expected mechanical behavior of FFF parts when subjected to important mechanical stresses \cite{Capote2017}. For these reasons, multiple attempts have been made to characterize the anisotropy of FFF manufactured objects, such as the studies performed by Koch \emph{et al.} \cite{Koch2017} and Rankouhi \emph{et al.} \cite{Rankouhi2016}, which show that the ultimate tensile strength of FFF coupons is sensitive to process parameters such as the layer thickness and, in particular, the orientation in which the plastic strands are laid during the build process -henceforth referred to as the bead orientation. Literature related to preventing failure through the use of \emph{Failure Criteria} (FC) in the design stages is scarce, given the difficulty of using commercially available FFF machines to produce test coupons with unconventional bead orientations, as well as the limitations inherent to development of failure envelopes. The large number of coupons required to properly characterize the failure behavior of parts is one of the main culprits for the small number of research articles on the topic. However, recent efforts include the developments of failure envelopes for \emph{Polyamide 12} (PA12) used in \emph{Selective Laser Sintering} (SLS) \cite{Obst2018}, and more importantly for this body of work, a failure surface for \emph{Acrylonitrile Butadiene Styrene} (ABS) used in FFF \cite{MazzeiCapote2019}. For the latter, certain test specimens in unconventional configurations had to be produced using a unique off-axis 3D printer developed in-house. In both cases, the researchers utilized a FC that incorporates stress interactions into the calculations of the failure surface, a feature that more recognized criteria, such as the Tsai-Wu model fail to take into account \cite{Osswald2017a}.

This research proposal springboards from the failure surface developed for FFF using ABS developed by Mazzei Capote \emph{et al}. \cite{MazzeiCapote2019}.

This work offers a comprehensive overview of AM technologies, FFF and shortcomings of current failure criteria in Chapter \ref{ch:bg}. Chapter \ref{ch:oocrit} details the failure criterion used throughout this work, as well as outlining its advantages over similar models. Chapters \ref{ch:exp} through \ref{ch:res} detail the experimental setup followed, as well as outlining noteworthy results. Finally, conclusions and recommendations are given in Chapter \ref{ch:concl} in the hopes of guiding future work on the topic. %REMEMBER TO MODIFY SPECIFICS OF THE TEXT HERE!.

%Nomenclature introduced in this chapter:
\nomenclature[A]{AM}{Additive Manufacturing}% 
\nomenclature[A]{RP}{Rapid Prototyping}%
\nomenclature[A]{CAD}{Computer Aided Design}%
\nomenclature[A]{FDM}{Fused Deposition Modeling\texttrademark}
\nomenclature[A]{FFF}{Fused Filament Fabrication}%
\nomenclature[A]{FC}{Failure Criterion}%
\nomenclature[A]{ABS}{Acrylonitrile Butadiene Styrene}%
\nomenclature[A]{PA12}{Polyamide 12}%

\end{document}